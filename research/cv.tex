\documentclass[12pt]{article}
\usepackage[whole]{bxcjkjatype}
\usepackage{epsf}
\usepackage{amsmath,amssymb}
\usepackage{bm}

\usepackage{graphicx}

\usepackage{comment}
\usepackage{multirow}
\usepackage{braket}

\usepackage{hyperref}
\usepackage{tgtermes}
\usepackage{float}

% set margin
\usepackage[margin=1in]{geometry}

% remove the page number
\pagestyle{empty}

% set section title all caps
\usepackage{titlesec}
\titleformat*{\section}{\normalsize\bf\MakeUppercase}
\titleformat*{\subsection}{\normalsize}
\titlespacing*{\section}{0pt}{0pt}{-.5em}
\titlespacing*{\subsection}{0pt}{0pt}{-1em}

% publications title
\renewcommand{\refname}{PUBLICATIONS\vspace*{1em}}

% base line stretch
\renewcommand{\baselinestretch}{1.2}

\allowdisplaybreaks[1]

% customize the title
\usepackage{titling}
\pretitle{\vspace{-1in}\begin{center}}
\title{{\Large So Chigusa}}
\posttitle{
  \par
  \vspace{1em}
  {\large Curriculum Vitae}
  \end{center}
}
\preauthor{\begin{center}}
\author{1 Cyclotron Rd, Berkeley, CA 94704 / (510)-486-4000 / \href{mailto:SoChigusa@lbl.gov}{SoChigusa@lbl.gov}}
\postauthor{\end{center}}
\date{\vspace{-.5in}}

\begin{document}
\maketitle

\section*{Education}
\begin{table}[h]
 \begin{tabular}{llll}
  2020 & Ph.D. in Physics, Department of Physics, University of Tokyo \\
  2017 & M.S. in Physics, Department of Physics, University of Tokyo \\
  2015 & B.S. in Physics, Department of Physics, University of Tokyo
 \end{tabular}
\end{table}

\section*{Professional appointments}
\begin{table}[h]
  \begin{tabular}{ll}
    2020 & Postdoctoral Fellow, Lawrence Berkeley National Laboratoy \\
    2020 & Postdoctoral Fellow, High Energy Accelerator Research Organization (KEK)
  \end{tabular}
\end{table}

\section*{Awards and Honors}
\begin{table}[h]
  \begin{tabular}{ll}
    2020 & Best presentation award for young scientists for Unraveling the History of the Universe 2020 \\ 2019 & Best Poster Award for HPNP 2019
  \end{tabular}
\end{table}

\section*{Grants and Fellowships}
\begin{table}[H]
  \begin{tabular}{ll}
    2020 & Research Fellowships for Young Scientists (PD), JSPS (3100000 JPY) \\
    2017 & Research Fellowships for Young Scientists (DC1), JSPS (2800000 JPY) \\
    2015 & Program for Leading Graduate Schools, MEXT
  \end{tabular}
\end{table}

\section*{Teaching experiences}
\begin{table}[H]
  \begin{tabular}{lp{6in}}
    2015 & Teaching Assistant for an undergraduate course ``Quantum Mechanics II'', Department of Physics, University of Tokyo, Apr.--Sep.
  \end{tabular}
\end{table}

\bibliographystyle{/Users/sochigusa/works/website/research/JHEP.bst}
\bibliography{/Users/sochigusa/works/website/research/publications.bib}
\nocite{*}

\section*{Invited Seminar Talks}
\begin{table}[H]\begin{tabular}{lp{6in}}2018 & ``Probing Electroweakly Interacting Massive Particles with Drell-Yan Process at 100 TeV Hadron Colliders'', Nagoya University, Oct 16 \\2019 & ``Solutions to Domain Wall Problem in Models with Discrete Flavor Symmetry'', Hokkaido University, Jan 11 \\2019 & ``Indirect Studies of Electroweakly Interacting Particles at 100 TeV Hadron Colliders'', KEK, Apr 9 \\2019 & ``Indirect Studies of Electroweakly Interacting Particles at 100 TeV Hadron Colliders'', Florida State University, May 10 \\2019 & ``Indirect Studies of Electroweakly Interacting Particles at 100 TeV Hadron Colliders'', University of Florida, May 16 \\2019 & ``Indirect Studies of Electroweakly Interacting Particles at 100 TeV Hadron Colliders'', Osaka University, Jul 23 \\2019 & ``Flowing to the Bounce'', Tohoku University, Oct 24 \\2020 & ``Detecting Light Boson Dark Matter through Conversion into Magnon (Online)'', TDLI and INPAC, May 14 \\2020 & ``Detecting Light Boson Dark Matter through Conversion into Magnon (Online)'', IBS, May 20 \\2020 & ``Detecting Light Boson Dark Matter through Conversion into Magnon (Online)'', Kyushu University, Jun 2 \\2020 & ``Detecting Light Boson Dark Matter through Conversion into Magnon (Online)'', UC Berkeley, Jun 12 \\2020 & ``Detecting Light Boson Dark Matter through Conversion into Magnon (Online)'', Nagoya University, Jun 22 \\2021 & ``固体中の「アクシオン」を用いた軽いボソン暗黒物質の直接探索'', Toyama, Kanazawa University, Mar 1 \\2022 & ``Quantum Simulations of Dark Sector Showers'', The University of Tokyo, May 23 \\2023 & ``Estimating eV-Scale Background Rates for Dark Matter Direct Detection'', KEK Theory Seminar, Sep 26 \\2023 & ``Vacuum Decay @ NLO: from the SM to a BSM for the muon g-2'', KEK IPNS Seminar, Sep 28 \\2023 & ``Light DM Search with Nitrogen-Vacancy Centers in Diamonds'', UC Davis, Nov 13 \\\end{tabular}\end{table}

\section*{Presentations at International Conferences}
\begin{table}[H]\begin{tabular}{lp{6in}}2016 & ``Bottom-Tau unification in Supersymmetric Model with Anomaly-Mediation'', SUSY 2016, Melbourne, Jul 5 \\2017 & ``Bottom-Tau Unification in Supersymmetric Models'', New Physics Forum, IPMU, Feb 6 \\2018 & ``Decay Rate of the Electroweak Vacuum in the Standard Model and Beyond'', Planck 2018, Bonn, May 24 \\2018 & ``Flavon Stabilization in Models with Discrete Flavor Symmetry'', KEK-PH 2018 winter, Tsukuba, Dec 6 \\2019 & ``Indirect Studies of Electroweakly Interacting Particles at 100 TeV Hadron Colliders'', Pheno 2019, Pittsburgh, May 6 \\2019 & ``Indirect Studies of Electroweakly Interacting Particles at 100 TeV Hadron Colliders'', SUSY 2019, Texas, May 22 \\2019 & ``Flowing to the Bounce'', NHWG26, Osaka, Aug 9 \\2019 & ``Indirect Studies of Electroweakly Interacting Particles at 100 TeV Hadron Colliders'', SI 2019, Gangneung, Korea, Aug 20 \\2020 & ``Flowing to the Bounce'', Berkeley Week, IPMU, Jan 14 \\2020 & ``Anomaly Mediation at Future Hadron Colliders'', KEK-PH 2020, Tsukuba, Aug 4 \\2021 & ``Direct detection of light bosonic dark matter using spin excitation'', COSMO'21, Online, Aug 3 \\2022 & ``Axion/Hidden-Photon Dark Matter Conversion into Condensed Matter Axion \textbf{(Invited)}'', KEK IPNS-IMSS-QUP Joint workshop, Online, Feb 8 \\2022 & ``Upper bound on the smuon mass from vacuum stability in the light of muon g−2 anomaly'', PPC 2022, St. Louis, Jun 7 \\2023 & ``Axion detection with spin dynamics: magnons and axions \textbf{(Invited)}'', Joint IQ Initiative and PITT PACC Workshop: Axions, Fundamental and Synthetic, The University of Pittsburgh, Apr 6 \\2023 & ``Topical theory talk: Vacuum stability \textbf{(Invited)}'', Workshop for Tera-Scale Physics and Beyond, Hakata, Jun 23 \\2023 & ``Light Dark Matter Search with Nitrogen-Vacancy Centers in Diamonds'', PASCOS 2023, UC Irvine, Jun 27 \\2023 & ``Light Dark Matter Search with Nitrogen-Vacancy Centers in Diamonds \textbf{(Invited)}'', PNU-IBS workshop on Axion Physics : Search for axions, Busan, Korea, Dec 6 \\\end{tabular}\end{table}

\subsection*{(Poster)}
\begin{table}[H]\begin{tabular}{lp{6in}}2019 & ``Probing Electroweakly Interacting Massive Particles with Precision Measurements at 100 TeV Hadron Colliders'', HPNP2019, Osaka, Feb 21 \\2022 & ``Roles of lattice defects in dark matter direct detection experiments'', QUPosium 2022, Tsukuba, Dec 13 \\\end{tabular}\end{table}

\section*{Presentations at Domestic Conferences}
\begin{table}[H]\begin{tabular}{lp{6in}}2016 & ``Bottom-Tau unification in Supersymmetric Model with Anomaly-Mediation'', JPS 2016, Miyazaki, Sep 21 \\2018 & ``Zero Mode Problem in the Calculation of Decay Rate of the SM Electroweak vacuum'', JPS 2018, Shinshu, Sep 15 \\2019 & ``Flavon Stabilization without Domain Wall Problem in Discrete Flavor Symmetry Models'', Neutrino Oscillation and Flavor Physics, Nagoya, Jun 11 \\2020 & ``マグノンを用いた軽いボソン暗黒物質の直接探索'', Unraveling the History of the Universe 2020, Online, Jun 2 \\2020 & ``特徴的なシグナルを用いた暗黒物質模型の探索 \textbf{(Invited)}'', 新テラスケール研究会, Online, Aug 11 \\2020 & ``XENON1T 実験の結果を説明する模型への制限'', ダークマターの懇談会2020 online, Online, Sep 8 \\2021 & ``スピン励起を用いた軽いボソン暗黒物質の直接探索 \textbf{(Invited)}'', KEK「素核宇・物性」連携研究会, Online, Mar 31 \\2021 & ``物性系を用いた軽い暗黒物質の直接探索 \textbf{(Invited)}'', PPP 2021, Online, Sep 9 \\2023 & ``高エネルギー反応におけるパートンシャワーへの量子計算の応用 \textbf{(Symposium talk)}'', JPS 2023 Spring, Online, Mar 25 \\2023 & ``LHC Run 3 と高輝度 LHC で探る新物理模型 \textbf{(Symposium talk)}'', JPS 2023 Fall, Tohoku University, Sep 18 \\\end{tabular}\end{table}

\subsection*{(Poster)}
\begin{table}[H]\begin{tabular}{lp{6in}}2017 & ``Bottom Tau Unification in Supersymmetric Models'', PPP 2017, Kyoto, Aug 3 \\2018 & ``Indirect Search of WIMP Dark Matter at Future 100 TeV Collider'', PPP 2018, Kyoto, Aug 9 \\2022 & ``レプトン加速器におけるヒッグス事象の機械学習を用いた事前選択'', PPP 2022, Online, Aug 29 \\\end{tabular}\end{table}

\section*{Service to Profession}
\vspace*{1em}
Journal manuscript review work of the following journals
\begin{itemize}
  \setlength\itemsep{0em}
  \item Physical Review Letters
  \item Physical Review D
  \item Journal of High Energy Physics
  \item Progress of Theoretical and Experimental Physics
\end{itemize}

\section*{Skills}
\begin{table}[H]
  \begin{tabular}{lp{6in}}
    Languages: & English -- Fluent, Japanese -- Native \\
    Computer: & c++, Mathematica, Python, LaTeX
  \end{tabular}
\end{table}

\end{document}
