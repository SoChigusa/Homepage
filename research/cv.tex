\documentclass[12pt]{article}
\usepackage[whole]{bxcjkjatype}
\usepackage{epsf}
\usepackage{amsmath,amssymb}
\usepackage{bm}

\usepackage{graphicx}

\usepackage{comment}
\usepackage{multirow}
\usepackage{braket}

\usepackage{hyperref}
\usepackage{tgtermes}

\setlength{\textwidth}{16.5cm}
\setlength{\textheight}{20.5cm}
\setlength{\oddsidemargin}{0cm}
\setlength{\evensidemargin}{0cm}
\setlength{\topmargin}{0cm}
\setlength{\footskip}{0cm}

\renewcommand{\arraystretch}{1.2}
\renewcommand{\baselinestretch}{1.1}

\renewcommand{\topfraction}{1.0}
\renewcommand{\bottomfraction}{1.0}

\renewcommand{\refname}{Publications}

\allowdisplaybreaks[1]

\title{\vspace{-2cm}\textbf{Curriculum Vitae}}
\author{So Chigusa}

\begin{document}
\large
\maketitle

\newcommand{\lsim}{\stackrel{<}{_\sim}}
\newcommand{\gsim}{\stackrel{>}{_\sim}}

\newcommand{\rem}[1]{{$\spadesuit$\bf #1$\spadesuit$}}

% \renewcommand{\theequation}{\thesection.\arabic{equation}}

\renewcommand{\thefootnote}{\arabic{footnote})}
\setcounter{footnote}{0}

\vspace{-5mm}
\subsection*{Personal Data}

\vspace{-3mm}

\begin{table}[h]
 \begin{tabular}{ll}
  First Name: & So % (颯)
      \\
  Last Name: & Chigusa % (千草)
      \\
  Date of Birth: & May 22, 1992 \\
  Place of Birth: & Kobe, Japan \\
  Nationality: & Japanese \\
  Age: & 31 \\
  Sex: & Male \\
 \end{tabular}
\end{table}

\vspace{-5mm}
\begin{table}[h]
 \begin{tabular}{ll}
  Affiliation: & Lawrence Berkeley National Laboratory \\
  Postcode: & 94720-8099 \\
  Address: & 1 Cyclotron Rd, Berkeley, CA \\
  Phone: & +1-510-486-4000 \\
  E-mail: &
      \href{mailto:SoChigusa@lbl.gov}{SoChigusa@lbl.gov}
      \\
  Webpage: & \url{https://website-sochigusa.vercel.app/} \\
 \end{tabular}
\end{table}
\vspace{-5mm}

\subsection*{Education}
\vspace{-3mm}
\begin{table}[h]
 \begin{tabular}{lll}
  \hline \hline
  Date & Degree & Institution \\ \hline
  Mar. 23, 2020 & Doctor of Philosophy (Physics) & University of Tokyo \\
  Mar. 23, 2017 & Master of Science (Physics) & University of Tokyo \\
  Mar. 25, 2015 & Bachelor of Science (Physics) & University of Tokyo \\
  \hline \hline
 \end{tabular}
\end{table}

\newpage
\subsection*{Professional experiences}
\vspace{-3mm}
\begin{table}[h]
\begin{tabular}{ll}
  \begin{tabular}{l}
    Sep. 2020 -- :
  \end{tabular} &
  \begin{tabular}{l}
    Postdoc, Lawrence Berkeley National Laboratoy
  \end{tabular}\\
  \begin{tabular}{l}
   Apr. 2020 -- Aug. 2022:
  \end{tabular} &
  \begin{tabular}{l}
   Postdoc, High Energy Accelerator Research Organization (KEK)
  \end{tabular}\\
  \begin{tabular}{l}
   Apr. 2015 -- Mar. 2020 :\\
  \quad
  \end{tabular} &
  \begin{tabular}{l}
   Ph.D. Student, Department of Physics, University of Tokyo\\
   (Dr. Takeo Moroi)
  \end{tabular}
 \end{tabular}
\end{table}
\vspace{-5mm}

\subsection*{Teaching experiences}
\vspace{-3mm}
\begin{table}[h]
 \begin{tabular}{ll}
  \begin{tabular}{l}
   Apr. 2015 -- Sep. 2015 :\\
   \quad
  \end{tabular} &
  \begin{tabular}{l}
   Teaching Assistant for Undergraduate Class ``Quantum Mechanics II''\\
   at Department of Physics, University of Tokyo
  \end{tabular}
 \end{tabular}
\end{table}
\vspace{-5mm}

\subsection*{Grants}
\vspace{-3mm}
\begin{table}[h]
 \begin{tabular}{ll}
  \begin{tabular}{l}
   Apr. 2020 -- Aug. 2022:\\
   \quad
  \end{tabular} &
  \begin{tabular}{l}
   JSPS, Research Fellowships for Young Scientists (PD)\\
   Amount: 3100000 JPY
  \end{tabular}\\
  \begin{tabular}{l}
   Apr. 2017 -- Mar. 2020 :\\
   \quad
  \end{tabular} &
  \begin{tabular}{l}
   JSPS, Research Fellowships for Young Scientists (DC1)\\
   Amount: 2800000 JPY
  \end{tabular}\\
  \begin{tabular}{l}
   Oct. 2015 -- Mar. 2020 :
  \end{tabular} &
  \begin{tabular}{l}
   MEXT, Program for Leading Graduate Schools
  \end{tabular}
 \end{tabular}
\end{table}
\vspace{-5mm}

\subsection*{Honors and Awards}
\begin{enumerate}
 \item Best presentation award for young scientists @ Unraveling the History of the Universe 2020, 2020/06/02,\item Best Poster Award @ HPNP 2019, 2019/02/22
\end{enumerate}

\bibliographystyle{/Users/sochigusa/works/website/research/JHEP.bst}
\bibliography{/Users/sochigusa/works/website/research/publications.bib}
\nocite{*}

\subsection*{Invited Seminar Presentations}
\begin{enumerate}
 \item ``Quantum Simulations of Dark Sector Showers'', 2022/05/23, The University of Tokyo,\item ``固体中の「アクシオン」を用いた軽いボソン暗黒物質の直接探索'', 2021/03/01, Toyama, Kanazawa University,\item ``Detecting Light Boson Dark Matter through Conversion into Magnon (Online)'', 2020/06/22, Nagoya University,\item ``Detecting Light Boson Dark Matter through Conversion into Magnon (Online)'', 2020/06/12, UC Berkeley,\item ``Detecting Light Boson Dark Matter through Conversion into Magnon (Online)'', 2020/06/02, Kyushu University,\item ``Detecting Light Boson Dark Matter through Conversion into Magnon (Online)'', 2020/05/20, IBS,\item ``Detecting Light Boson Dark Matter through Conversion into Magnon (Online)'', 2020/05/14, TDLI and INPAC,\item ``Flowing to the Bounce'', 2019/10/24, Tohoku University,\item ``Indirect Studies of Electroweakly Interacting Particles at 100 TeV Hadron Colliders'', 2019/07/23, Osaka University,\item ``Indirect Studies of Electroweakly Interacting Particles at 100 TeV Hadron Colliders'', 2019/05/16, University of Florida,\item ``Indirect Studies of Electroweakly Interacting Particles at 100 TeV Hadron Colliders'', 2019/05/10, Florida State University,\item ``Indirect Studies of Electroweakly Interacting Particles at 100 TeV Hadron Colliders'', 2019/04/09, KEK,\item ``Solutions to Domain Wall Problem in Models with Discrete Flavor Symmetry'', 2019/01/11, Hokkaido University,\item ``Probing Electroweakly Interacting Massive Particles with Drell-Yan Process at 100 TeV Hadron Colliders'', 2018/10/16, Nagoya University
\end{enumerate}

\subsection*{Presentations at International Conferences}
\subsubsection*{(Oral)}
\begin{enumerate}
 \item ``Light Dark Matter Search with Nitrogen-Vacancy Centers in Diamonds'', 2023/06/27, PASCOS 2023, UC Irvine,\item ``Topical theory talk: Vacuum stability'', 2023/06/23, Workshop for Tera-Scale Physics and Beyond, Hakata,\item ``Axion detection with spin dynamics: magnons and axions (Invited)'', 2023/04/06, Joint IQ Initiative & PITT PACC Workshop: Axions, Fundamental and Synthetic, The University of Pittsburgh,\item ``Upper bound on the smuon mass from vacuum stability in the light of muon g−2 anomaly'', 2022/06/07, PPC 2022, St. Louis,\item ``Axion/Hidden-Photon Dark Matter Conversion into Condensed Matter Axion (Invited)'', 2022/02/08, KEK IPNS-IMSS-QUP Joint workshop, Online,\item ``Direct detection of light bosonic dark matter using spin excitation'', 2021/08/03, COSMO'21, Online,\item ``Anomaly Mediation at Future Hadron Colliders'', 2020/08/04, KEK-PH 2020, Tsukuba,\item ``Flowing to the Bounce'', 2020/01/14, Berkeley Week, IPMU,\item ``Indirect Studies of Electroweakly Interacting Particles at 100 TeV Hadron Colliders'', 2019/08/20, SI 2019, Gangneung, Korea,\item ``Flowing to the Bounce'', 2019/08/09, NHWG26, Osaka,\item ``Indirect Studies of Electroweakly Interacting Particles at 100 TeV Hadron Colliders'', 2019/05/22, SUSY 2019, Texas,\item ``Indirect Studies of Electroweakly Interacting Particles at 100 TeV Hadron Colliders'', 2019/05/06, Pheno 2019, Pittsburgh,\item ``Flavon Stabilization in Models with Discrete Flavor Symmetry'', 2018/12/06, KEK-PH 2018 winter, Tsukuba,\item ``Decay Rate of the Electroweak Vacuum in the Standard Model and Beyond'', 2018/05/24, Planck 2018, Bonn,\item ``Bottom-Tau Unification in Supersymmetric Models'', 2017/02/06, New Physics Forum, IPMU,\item ``Bottom-Tau unification in Supersymmetric Model with Anomaly-Mediation'', 2016/07/05, SUSY 2016, Melbourne
\end{enumerate}
\subsubsection*{(Poster)}
\begin{enumerate}
 \item ``Roles of lattice defects in dark matter direct detection experiments'', 2022/12/13, QUPosium 2022, Tsukuba,\item ``Probing Electroweakly Interacting Massive Particles with Precision Measurements at 100 TeV Hadron Colliders (poster)'', 2019/02/21, HPNP2019, Osaka
\end{enumerate}

\subsection*{Presentations at Domestic Conferences}
\subsubsection*{(Oral)}
\begin{enumerate}
 \item ``高エネルギー反応におけるパートンシャワーへの量子計算の応用(シンポジウム講演)'', 2023/03/25, JPS 2023 Spring, Online,\item ``物性系を用いた軽い暗黒物質の直接探索(招待講演)'', 2021/09/09, PPP 2021, Online,\item ``スピン励起を用いた軽いボソン暗黒物質の直接探索(招待講演)'', 2021/03/31, KEK「素核宇・物性」連携研究会, Online,\item ``XENON1T 実験の結果を説明する模型への制限'', 2020/09/08, ダークマターの懇談会2020 online, Online,\item ``特徴的なシグナルを用いた暗黒物質模型の探索(招待講演)'', 2020/08/11, 新テラスケール研究会, Online,\item ``マグノンを用いた軽いボソン暗黒物質の直接探索'', 2020/06/02, Unraveling the History of the Universe 2020, Online,\item ``Flavon Stabilization without Domain Wall Problem in Discrete Flavor Symmetry Models (in Japanese)'', 2019/06/11, Neutrino Oscillation and Flavor Physics, Nagoya,\item ``Zero Mode Problem in the Calculation of Decay Rate of the SM Electroweak vacuum'', 2018/09/15, JPS 2018, Shinshu,\item ``Bottom-Tau unification in Supersymmetric Model with Anomaly-Mediation'', 2016/09/21, JPS 2016, Miyazaki
\end{enumerate}
\subsubsection*{(Poster)}
\begin{enumerate}
 \item ``レプトン加速器におけるヒッグス事象の機械学習を用いた事前選択(ポスター)'', 2022/08/29, PPP 2022, Online,\item ``Indirect Search of WIMP Dark Matter at Future 100 TeV Collider (Poster)'', 2018/08/09, PPP 2018, Kyoto,\item ``Bottom Tau Unification in Supersymmetric Models (Poster)'', 2017/08/03, PPP 2017, Kyoto
\end{enumerate}

% \subsection*{Poster Presentations at International Summer Schools}
% \begin{enumerate}
%  % Summer School here
% \end{enumerate}

\end{document}
