\documentclass[12pt]{article}
\usepackage[whole]{bxcjkjatype}
\usepackage{epsf}
\usepackage{amsmath,amssymb}
\usepackage{bm}

\usepackage{graphicx}

\usepackage{comment}
\usepackage{multirow}
\usepackage{braket}

\usepackage{hyperref}
\usepackage{tgtermes}
\usepackage{float}
\usepackage{xcolor}
\usepackage{siunitx}
\usepackage{mhchem}
\usepackage{cite}

% set margin
\usepackage[margin=1in]{geometry}

% remove the page number
% \pagestyle{empty}

% set section title all caps
\usepackage{titlesec}
\titleformat*{\section}{\large\bf}
\titleformat*{\subsection}{\normalsize\bf}
\titlespacing*{\section}{0pt}{2em}{1em}
\titlespacing*{\subsection}{0pt}{2em}{1em}

% publications title
\renewcommand{\refname}{References}

% base line stretch
\renewcommand{\baselinestretch}{1.1}

\allowdisplaybreaks[4]

% customize the title
\usepackage{titling}
\pretitle{\vspace{-1in}}
\title{{\Large Research Statement}}
\posttitle{
  \par
}
\preauthor{\vspace{0.5em}}
\author{
  \indent{\large So Chigusa}\\
  \indent\textit{sochigusa@lbl.gov}\\
  \indent\textit{Lawrence Berkeley National Laboratory, 1 Cyclotron Rd, Berkeley, CA 94704, USA}
}
\postauthor{}
\date{\vspace{-3em}}

\begin{document}
\maketitle

Nowadays, the improvement of technologies in quantum science is so rapid that various new approaches to problems are becoming available in a short period.
These new approaches include \textbf{quantum sensing} techniques to detect a faint signal and \textbf{quantum computation} that allows us to simulate the dynamics of quantum systems by directly manipulating quantum states.
Given this situation, it is incumbent on theoretical particle physicists to stay in touch with new technologies and to develop ways to use them in the search for beyond the standard model physics.
I have been working along this line; I mainly study the phenomenology of the standard model and beyond, including the \textbf{dark matter} search and important quantum corrections to the \textbf{parton shower} process using various quantum technologies. 
My main research directions are categorized as \textit{(i) direct detection of light dark matter with quantum sensing techniques} and \textit{(ii) construction of quantum algorithms applicable to systems such as parton shower}.
In addition, as \textit{(iii) other directions}, I work on several different approaches to further explore beyond the standard model physics, including \textbf{false vacuum decay} rate calculation and \textbf{collider phenomenology}.

\subsection*{Direct detection of light dark matter with quantum sensing techniques}

The conventional dark matter direct detection programs, which mainly focus on the $\mathrm{GeV}$-mass region, give us no sign of dark matter so far.
This motivates the community to explore different dark matter mass window, which is widely open, especially towards the lighter region.
However, direct detection of light dark matter is challenging because of the low excitation energy and the small event rate.
Quantum sensing techniques are useful to detect such a faint signal and also to amplify it to improve the sensitivity.

If the dark matter has a wavelength longer than the interatomic spacing, the dark matter collision with a material excites collective modes such as phonons and magnons.
Along this line, I worked on three different approaches using different collective excitation signal of spins, i.e., the \textbf{magnon excitation} \cite{Chigusa:2020gfs}, the \textbf{axion-like excitation} \cite{Chigusa:2021mci}, and the \textbf{nuclear magnon excitation} \cite{Chigusa:2023hmz}, each of which probe different dark matter couplings.
Many ongoing experiments are searching for spin excitations with the same or similar ideas, which might discover the dark matter.
In addition, as part of an attempt to accurately identify the background events of many ongoing experiments based on phonons, I studied the \textbf{lattice defects} as a possible source of $\mathrm{eV}$-scale background events \cite{Frenkel}.

In this context, more specialized quantum sensing techniques are useful to develop a new approach with different target frequency or to enhance the dark matter signal.
In Ref. \cite{Chigusa:2023hms}, I proposed the light dark matter search based on the \textbf{nitrogen-vacancy center} magnetometry, which has a broad frequency coverage.
Recently, an experiment based on this idea has been launched in the International Center for Quantum-field Measurement Systems for Studies of the Universe and Particles (QUP).
I also studied the possibility of enhancing the nuclear-magnon signal excited in superfluid $\mathrm{^3He}$ with \textbf{squeezing} and revealed the condition for sensitivity improvement \cite{Chigusa:2023szl}.

Similarly to \cite{Chigusa:2023szl}, squeezing of the spin state could have potential to enhance sensitivities of the setups using spin excitation considered in \cite{Chigusa:2020gfs}, \cite{Chigusa:2021mci}, \cite{Chigusa:2023hmz}.
Thus, it is useful to consider what kind of measurement can be performed to profit from this approach.
Another important ingredient of quantum sensing techniques is \textbf{entanglement}.
If $N$ qubits are available for measurement, entanglement allows the event rate to scale ideally as $\propto N^2$, contrary to the naive anticipation $\propto N$.
Since the recent technologies have already achieved entanglement among multiple qubits, it is an important task to think of the best setup with entanglement taking account of non-trivial effects on experimental parameters such as increased decoherence time.
New ideas should be proposed not only to improve sensitivity but also to explore a \textbf{broad mass range} or to explore \textbf{various dark matter couplings}.
Finally, similar ideas are applicable to other targets including cosmic axion background and high-frequency gravitational waves, as partly discussed in \cite{Chigusa:2023bga}.

\subsection*{Construction of quantum algorithms applicable to systems such as parton shower}

Nowadays, quantum computing resources with a sizable number of qubits are open to the public.
Combined with the rapid improvement of technologies that promises the near-future potential of quantum computers as a tool to investigate the dynamics of quantum systems, now is the time to study possible quantum algorithms we can equip on the current and near-future quantum computers.

Parton shower algorithm is a classical approach to computing the multi-emission cross sections, which has been widely used for collider and astroparticle physics simulations.
However, because the algorithm is based on the classical probability distribution, important \textbf{quantum interference effects} are not incorporated under the existence of a non-trivial flavor structure of fermions, which could significantly modify the particle multiplicity distribution \cite{Chigusa:2022act}.
In addition, due to the flavor index assignment, the number of diagrams we need to calculate grows exponentially as particle multiplicity increases, and the computational cost of classical calculation also increases exponentially.
Motivated by this, I investigated a \textbf{quantum parton shower algorithm} using a veto procedure.
This is the first quantum algorithm that operates on polynomial computational resources while accounting for quantum interference effects and reconstructing full kinematics information \cite{Bauer:2023ujy}.

The quantum veto algorithm studied in \cite{Bauer:2023ujy} still has a large room for improvement in both computational and physical aspects.
Although the evolution variable of parton shower, the virtuality, is discretized in the current algorithm, one can in principle sample the virtuality by using the exponentiated probability density as is done in classical parton shower algorithms, which speeds up the simulation further.
Also, a physically more involved algorithm with soft interference could be constructed if the history information, the list of partons that emit at each step, is stored in qubits.
Finally, this research project can be extended to simulation of non-perturbative dynamics, which is often difficult to treat in conventional methods.
When direct observation of the dynamics is difficult, such as in the case of false vacuum decay, the quantum simulation itself could serve as a proof of concept.

It is also important to combine quantum algorithms with quantum sensing to, e.g., improve the sensitivity to the light dark matter signal.
One example of the possibly useful algorithms is the \textbf{amplitude amplification}, which allows the probability for signal detection to scale ideally as $\propto N^2$ with the $N$ repeated measurements.

\subsection*{Other directions}

The observed value of the Higgs mass $M_h \simeq 125\,\mathrm{GeV}$ results in the Higgs self-coupling $\lambda$ running to negative according to the standard model renormalization group flow, which indicates that the electroweak vacuum of the standard model is not absolutely stable.
Whether the lifetime of the electroweak vacuum is longer than the age of the universe or not, the so-called \textbf{electroweak vacuum stability}, should be properly judged to test if the physics of the standard model and beyond is compatible with our universe.
On the other hand, false vacuum decay rate calculation at the next-to-leading order tends to contain complicated differential equations that are computationally hard to solve and divergences sourced from symmetries that are conceptually difficult to treat.

In \cite{Chigusa:2017dux}, \cite{Chigusa:2018uuj}, I overcame these technical difficulties and obtained a semi-analytic one-loop expression of the electroweak vacuum decay rate in the standard model and beyond without additional Higgs bosons.
This result allows us to calculate the decay rate much faster and more precisely than in the past.
I generalized this result in \cite{Chigusa:2020jbn} and provided the first semi-analytic expression of the one-loop vacuum decay rate in the general gauge theory.
I used this expression to test the electroweak vacuum decay in a setup of the minimal supersymmetric standard model that can resolve the possible muon $g-2$ tension and obtained an upper bound on the mass of certain new particles \cite{Chigusa:2022xpq}, \cite{Chigusa:2023mqy}; a part of the allowed mass range can be searched for by the future collider experiments.

One of the recently evolving computational tools is \textbf{machine learning}, which is suitable for analyzing the huge and complicated data of collider experiments.
I adopted this tool to achieve better preselection of the Higgs boson at future lepton colliders, which will be the basis technology for the Higgs study program.
I compared the performance of this approach with that of the traditional boosted decision tree approach and concluded that $\mathcal{O}(10)\,\%$ improvement can be expected \cite{Chigusa:2022svv}.
Also, a large amount of effort has been devoted to realizing the muon collider, which has the potential to reach higher energy than electron-positron colliders, while maintaining cleaner signals than hadron colliders.
I explored a way to look for a new particle that couples dominantly to top quarks through four-top events and the resonance peak search in the muon collider.
I showed that the tendency of having more \textbf{boosted top jets} is a clear advantage of the high-energy muon collider against the large hadron collider, despite the challenges such as the beam-induced background \cite{Chigusa:2023rrz}.

\subsection*{Conclusion}

The field of quantum science is now in an interesting time with many new experimental technologies and theoretical ideas coming out daily.
I intend to take full advantage of the opportunity to develop new ideas for phenomenological studies of particle physics based on the rapid developments in quantum science, aiming to unravel the nature of the universe.

\bibliographystyle{/Users/sochigusa/works/website/research/JHEP.bst}
\bibliography{publications.bib}

\end{document}
